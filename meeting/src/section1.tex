\section{Tecnologie analizzate}
Sono stati analizzati i seguenti framework frontend JavaScript:
\begin{itemize}
    \item Angular
    \item React
\end{itemize}
Si è inoltre presa analisi di un framework backend con Python:
\begin{itemize}
    \item Django
\end{itemize}

\subsection{Pro e contro}
\subsubsection{Django}
Pro:
\begin{itemize}
    \renewcommand\labelitemi{\texttt{+}}
    \item Il framework è basato su Python, sarà più semplice implementare il modulo di IA
    \item Struttura molto simile al modello MVC studiato in TSW
    \item Meccanismo simile alle pagine JSP utilizzate in TSW
    \item Ottimo per progetti grandi e scalabili
    \item Ottime funzionalità di gestione, tra cui in particolare un interfaccia web 
            che permette l'interazione col DB
\end{itemize}
Contro:
\begin{itemize}
    \renewcommand\labelitemi{\texttt{-}}
    \item Difficoltà di utilizzo per principianti
    \item Probabilmente andrebbe comunque integrato con un framework frontend
\end{itemize}

\subsubsection{Angular}
Pro:
\begin{itemize}
    \renewcommand\labelitemi{\texttt{+}}
    \item Robusto per progetti di grande spessore e grandi dimensioni
    \item Ottimo per gestire tanti dati
    \item Disponibilità di molte librerie frontend per scrivere UI di buona fattura
    \item Permette d'inserire codice JavaScript all'interno di HTML, similmente alle pagine JSP studiate in TSW
    \item Semplicità di organizzazione: i progetti Angular presentano quasi sempre la stessa struttura
    \item Autogestito: funzionalità out of the box
    \item Piace a Luca
\end{itemize}
Contro:
\begin{itemize}
    \renewcommand\labelitemi{\texttt{-}}
    \item Richiede una discreta quantità di tempo per poter imparare le funzionalità essenziali
    \item L'avere molte librerie e funzioni lo rende abbastanza dispersivo e disorientante
\end{itemize}

\subsubsection{React}
Pro:
\begin{itemize}
    \renewcommand\labelitemi{\texttt{+}}
    \item Grande comunity: è il framework JavaScript più popolare sul mercato
    \item Minimale: di facile apprendimento e veloce nella scrittura
    \item Veloce nel permettere di aggiungere feature aggiuntive al progetto
    \item Ottime librerie UI (analogamente ad Angular)
\end{itemize}
Contro:
\begin{itemize}
    \renewcommand\labelitemi{\texttt{-}}
    \item Funzionamento inverso alle pagine JSP: il codice HTML viene inserito nei file JavaScript (JSX)
    \item Va abbinato a librerie per poterlo utilizzare in modo efficiente
\end{itemize}
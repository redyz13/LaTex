\begin{multicols*}{2}
[\section*{Blocco 2}]

\subsection*{Variabili aleatorie}
\begin{itemize}
    \item $\{\omega \in \Omega : X(\omega) \in B\} \in \mathscr{F}$
    \item $\{\omega \in \Omega : X(\omega) \le x\} \in \mathscr{F}$
\end{itemize}

\subsection*{Funzione di distribuzione}
$$
F_X(x) = P(X \le x) = P(\{\omega \in \Omega : X(\omega) \le x\})
$$
Altre probabilità che $x$ appartenga a un qualsiasi Boreliano:
\begin{enumerate}
    \item $P(X < x) = F_X(x^-)$
    \item $P(X = x) = F_X(x) - F_X(x^-)$
    \item $P(x_1 < X < x_2) = F_X(x_2^-) - F_X(x_1)$
    \item $P(x_1 < X \le x_2) = F_X(x_2) - F_X(x_1)$
    \item $P(x_1 \le X < x_2) = F_X(x_2^-) - F_X(x_1^-)$
    \item $P(x_1 \le X \le x_2) = F_X(x_2) - F_X(x_1^-)$
    \item $P(X > x) = 1 - F_X(x)$
    \item $P(X \ge x) = 1 - F_X(x^-)$
\end{enumerate}

\subsection*{Variabili aleatorie discrete}
$$
P(X \in S_x) = \sum_{r : x_r \in S_x}P(X = X_r) = 1
$$
$P(X = x_r)$ definisce la funzione di probabilità
$$
p_X(x) = P(X = x) =
\begin{cases}
    p_n & x = x_n (n = 1,2,\dots)\\
    0 & \text{altrimenti}
\end{cases}
$$
La funzione di distribuzione può essere espressa come
$$
F_X(x) = P(X \le x) = \sum_{n: x_n \le x} pn \ (x \in \mathbb{R})
$$

\subsection*{Variabili aleatorie assolutamente continue}
$$F_X(x) = \int_{-\infty}^{x} f_X(y)dy$$
$f_X(y) = $ densità di probabilità
$$f_X(x) = \frac{dF_X(x)}{dx}$$

\subsubsection*{Proprietà}
\begin{itemize}
    \item $P(X = x) = 0$
    \item $f_X(x) \ge 0$
    \item $\displaystyle \int_{-\infty}^{+\infty}f_X(x)dx=1$
    \item $P(X > x) = \displaystyle \int_{x}^{+\infty}f_X(x)dx$
    \item $P(a \le X \le b) = \displaystyle \int_{a}^{b}f_X(x)dx=F_X(b)-F_X(a)$
    \item $P(X \in B) = \displaystyle \int_{B}f_X(z)dz$
\end{itemize}

\subsection*{Valore medio}
Variabile aleatoria discreta:
$$
E(X) = \sum_{r: x_r \in S}x_r p_X(x_r)
$$
Variabile aleatoria assolutamente continua:
$$
E(X) = \int_{-\infty}^{+\infty}x f_X(x)dx
$$

\subsection*{Momenti di una variabile aleatoria}
Momento di ordine $n$:
$$
\mu_n = E(X^n) \implies \mu_n=\sum_{r: x_r \in S}x_r^n p_X(x_r)
$$
Momento centrale:
\begin{align*}
\overline{\mu_n} = E[(X-E(X))^n] \implies
\\\overline{\mu_n}=\sum_{r: x_r \in S}(x_r-\mu_1)^n p_X(x_r)
\end{align*}

\subsection*{Varianza}
\begin{align*}
\mathit{Var}(x) = E(X^2) - [E(X)]^2 =
\\\sum_{r: x_r \in S}(x_r - \underbrace{\mu_1}_{\text{media}})^2 p_X(x_r)
\end{align*}

\end{multicols*}

\newpage
\subsection*{Variabili aleatorie discrete (distribuzione uniforme)}
Sia $x = 1,2,\dots,n$ tutti con la stessa probabilità $X \sim \mathscr{U}_d(n) $
\begin{align*}
    P_X(x) &=
    \begin{cases}
        \frac{1}{n} & x = 1,2,\dots,n\\
        0 & \text {altrimenti}
    \end{cases}
    \\
    F_X(x) = P(X \le x) &=
    \begin{cases}
        0 & x < 1\\
        \frac{k}{n} & k \le x < k + 1 \ (k = 1,2,\dots,n-1)\\
        1 & x \ge n
    \end{cases}
\end{align*}
Se $n = 1$ allora è una distribuzione degenere e vale
$$
E(X) = \frac{n+1}{2}
$$
$$
\mathit{Var}(X) = \frac{n^2-1}{12}
$$
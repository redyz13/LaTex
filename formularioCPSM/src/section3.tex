\begin{multicols*}{2}
[\section*{Blocco 3}]
    
\subsection*{Vettori aleatori}
\begin{itemize}
    \item $\{\omega \in \Omega : (X_1(\omega),\dots,X_n(\omega)) \in B\} \in \mathscr{F}$
    \item $\{\omega \in \Omega : (X_1(\omega) \le x_1, \dots, X_n(\omega) \le x_n)\}
        = \displaystyle \bigcap_{i=1}^{n}\{\omega \in \Omega : X_1(\omega) \le x_i\} \in \mathscr{F}$
\end{itemize}

\subsection*{Funzione di distribuzione congiunta o marginale}
Sia $X = (X_1,\dots,X_n)$ un vettore aleatorio $X: \Omega \to \mathbb{R}^n$ 
\begin{align*}
&F_{X_1,\dots,X_n}(X_1,\dots,X_n) = \\
&P(X_1 \le x_1, \dots, X_n \le x_n) = \\
&P(\{\omega \in \Omega : X_1(\omega) \le x_1, \dots, X_n(\omega) \le x_n\})\\
\end{align*}

\subsection*{Vettori aleatori}
Consideriamo un vettore aleatorio bidimensionale
$$
F_{X,Y} = P(X \le x, Y \le y) \ \forall \ x,y \in \mathbb{R}^2
$$

\subsection*{Vettori aleatori discreti}
\begin{align*}
\exists \ S = S_1 \times S_2, \text{ con } &S_1 = \{x_1,\dots,x_n\},
\\&S_2 = \{y_1,\dots,y_n\}
\end{align*}

$$
P[(X,Y) \in S] = 1\\
$$

\begin{align*}
\{ \omega \in \Omega : (X(\omega), Y(\omega)) \in S\} = 
    \bigcup_{\{i: x_i \in S_1\}}
    \\\bigcup_{\{j: y_j \in S_2\}}
    \{\omega \in \Omega : X(\omega) = x_i, Y(\omega) = y_j\}\\
\text{ quindi per l'incompatibilità } =
    \sum_{\{i: x_i \in S_1\}}
    \\\sum_{\{j: y_j \in S_2\}}
    P(X = x_i, Y = y_j) = 1
\end{align*}

\subsubsection*{Funzione di probabilità congiunta}

$$p_{X,Y}(x, y) =
\begin{cases}
    p_{i,j} & x = x_i, y = y_j \ (i, j = 1,2,\dots)\\
    0 & \text{altrimenti}
\end{cases}$$

$$
P\{(X,Y) \in D\} = \sum_{(x,y) \in D}P_{X,Y}(x,y)
$$

$$
F_{X,Y}(x,y) = \sum_{\{i : x_i \le x\}} \sum_{\{j : y_j \le j\}} p_{i,j} 
\ \forall \ (x, y) \in \mathbb{R}^2
$$

\subsubsection*{Funzioni di probabilità marginali}
$$
\{Y=y\} = \bigcup_{i : x_i \in S_1}\{X=x_i, Y=y\}
$$

$$
\{X=x\} = \bigcup_{j : y_j \in S_2}\{X=x, Y=y_j\}
$$

$$
p_Y(y) = \sum_{\{i: x_i \in S_1\}}p_{X,Y}(x_i, \underbrace{y}_{\text{fissata}}) \ \forall \ y \in \mathbb{R}
$$

$$
p_X(x) = \sum_{\{j: y_j \in S_2\}}p_{X,Y}(\underbrace{x}_{\text{fissata}}, y_j) \ \forall \ x \in \mathbb{R}
$$

Da $P_{X,Y}(x,y)$ ricaviamo le due funzioni e non viceversa

\subsection*{Vettori aleatori assolutamente continui}
$$
F_{X,Y}(x,y) = \int_{-\infty}^{x}du \int_{-\infty}^{y} f_{X,Y}(u,v) dv
$$

$$
f_{X,Y}(x,y) = \frac{\delta^2}{\delta_x dy} F_{X,Y}(x,y)
$$

\subsubsection*{Proprietà}
$$
f_{X,Y}(x,y) \ge 0 \ \forall \ (x,y) \in \mathbb{R}^2
$$

$$
\int_{-\infty}^{+\infty}du \int_{-\infty}^{+\infty} f_{X,Y}(u,v) dv = 1
$$

$$
P\{(X,Y) \in D\} = \int_{D} f_{X,Y}(u,v) dv
$$

\subsubsection*{Densità di probabilità marginali}
$$
F_X(x) = \int_{-\infty}^{x}\left[\int_{\infty}^{+\infty} f_{X,Y}(u,v) dv\right] du \ \forall \ x \in \mathbb{R}
$$

$$
F_Y(y) = \int_{-\infty}^{y}\left[\int_{\infty}^{+\infty} f_{X,Y}(u,v) du\right] dv \ \forall \ y \in \mathbb{R}
$$

$$
f_X(x) = \int_{-\infty}^{+\infty} f_{X,Y}(x,v) dv \ \forall \ x \in \mathbb{R}
$$

$$
f_Y(y) = \int_{-\infty}^{+\infty} f_{X,Y}(u,y) du \ \forall \ y \in \mathbb{R}
$$

\subsection*{Indipendenza vettori aleatori}
Siano $X_1,\dots,X_n$ si ha
\begin{align*}
P(X_1 \in B,\dots, X_n \in B_n) =
\\P(X_1 \in B_1) \cdot \ \dots \ \cdot P(X_n \in B_n)
\end{align*}
\begin{align*}
F_{X_1}, \dots, _{X_n}(x_1, \dots, x_n) &=
\\P(X_1 \le x_1, \dots, X_n \le x_n) &= 
\\P(X_1 \le x_1)\cdot \ \dots \ \cdot P(X_n \le x_n) &= 
\\F_{X_1}(x_1) \cdot \ \dots \ \cdot F_{X_N}(x_n)
\end{align*}

\subsubsection*{Indipendenza vettori aleatori discreti}
$$
p_{X_1},\dots,_{X_n}(x_1,\dots,x_n) = p_{X_1}(x_1) \cdot \ \dots \ \cdot p_{X_n}(x_n)
$$

\subsubsection*{Indipendenza vettori aleatori continui}
$$
f_{X_1},\dots,_{X_n}(x_1,\dots,x_n) = f_{X_1}(x_1) \cdot \ \dots \ \cdot f_{X_n}(x_n)
$$

\subsection*{Valore medio e momento}
Sia $X= (X_1,\dots,X_n)$ una variabile aleatoria e $g: \mathbb{R}^n \to \mathbb{R}, \ Y = g(X)$
\begin{enumerate}
    \item Se $X$ è discreta
    \begin{align*}
    E[g(x)] = \sum g(x)p_X(x) =
    \\\sum_{x_1 \in S_1} \dots \sum_{x_n \in S_n}
    g(x_1,\dots,x_n) \cdot
    \\\cdot P(X_1=x_1,\dots,X_n=x_n)
    \end{align*}
    \item Se $X$ è assolutamente continua
    \begin{align*}
    E[g(x)] = \int_{\mathbb{R}} g(x)f_x(x) dx =
    \\\int_{-\infty}^{+\infty} \dots \int_{-\infty}^{+\infty}
    g(x_1,\dots,x_n) \cdot
    \\\cdot f_{X_1},\dots,_{X_n}(x_1,\dots,x_n)
    \end{align*}
\end{enumerate}

\subsection*{Teorema}
$X = (X_1,\dots,X_n) \land c_1,\dots,c_n \in \mathbb{R} \implies$
$E(c_1 X_1 + \dots + c_n X_n) = c_1 E(X_1) + \dots + c_n E(X_n)$
\begin{itemize}
    \item Se le componenti $X = X_1, \dots, X_n$ sono indipendenti
    \begin{itemize}
        \item $E(X_1,\dots,X_n) = E(X_1) \cdot \ \dots \ \cdot E(X_n)$
        \item $\mathit{Var}(X_1,\dots,X_n) = E(X_1^2) \cdot \ \dots \ \cdot E(X_n^2)
        - [E(X_1)]^2 \cdot \ \dots \ \cdot [E(X_n)]^2
        $
    \end{itemize}
\end{itemize}

\subsection*{Momento misto di ordine $i,j$}
$$
\mu_{i,j} = E(X^i Y^j)
$$

\subsection*{Momento misto centrale di ordine $i,j$}
$$
\overline{\mu_{i,j}} = E[(X- E(X))^i (Y - E(Y))^j]
$$
\begin{itemize}
    \item Se $i = 0$
    \begin{itemize}
        \item $\mu_{0,j} = E(Y^j), \ \overline{\mu_{0,j}} = E[(Y - E(Y))^j]$
    \end{itemize}
    \item Se $j = 0$
    \begin{itemize}
        \item $\mu_{i,0} = E(X^i), \ \overline{\mu_{i,0}} = E[(X - E(X))^i]$
    \end{itemize}
\end{itemize}

\subsection*{Covarianza}
$$
\mathit{Cov}(X,Y) = \overline{\mu_{1,1}} = E[(X - E(X)) - (Y - E(Y))]
$$
 
Sia $(X,Y)$ un vettore aleatorio
\begin{itemize}
    \item $\mathit{Cov}(X,Y) = \mathit{Cov}(Y,X)$
    \item $\mathit{Cov}(aX + b, cY + d) = ac \mathit{Cov}(X,Y)$
    \item $\mathit{Cov}(X,Y) = E(XY) - E(X)E(Y)$
\end{itemize}

Se $(X,Y)$ sono indipendenti allora $\mathit{Cov}(X,Y) = 0$ (non vale il viceversa)

\subsection*{Coefficiente di correlazione}
$$
\varrho(X,Y) = \frac{\mathit{Cov}(X,Y)}{\sqrt{\mathit{Var}(X) \mathit{Var}(Y)}}
= \frac{\mathit{Cov}(X,Y)}{\sigma_x \sigma_y}
$$
$$
|\varrho(X,Y)| \le 1
$$

\subsection*{Valore medio utile}
\begin{align*}
E(X_1 \cdot \ \dots \ \cdot X_n) = \sum_{x_1 \in S_1} \dots 
\\\sum_{x_n \in S_n}
P(X_1=x_1, \dots, X_n=x_n)
\end{align*}

\end{multicols*}
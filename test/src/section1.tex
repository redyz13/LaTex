\section{Ricerca}
Trovare un elemento $k$ in un array 
ordinato in tempo $O(\log n)$ tramite il paradigma Divide et Impera.

\subsubsection*{Relazione di ricorrenza:}

$$T(n) \le
\begin{cases}
    c_0 & \text{se $n \le 1$ } \text{oppure $k$ è l'elemento centrale}\\
    T(n/2) + c & \text{altrimenti}
\end{cases} $$

\subsubsection*{Algoritmo:}

\begin{algorithm}[H]
\caption{Ricerca binaria ricorsiva}
\label{alg1}
\begin{algorithmic}[1]
\STATE RicercaBinariaRicorsiva(a, k, sinistra, destra):
\begin{ALC@g}
\IF{(sinistra $>$ destra)}
    \RETURN -1
\ENDIF
\STATE c = (sinistra + destra) / 2
\IF{k == a[c]}
    \RETURN c
\ENDIF
\IF{sinistra == destra}
    \RETURN -1
\ENDIF
\IF{(k $<$ a[c])}
    \RETURN RicercaBinariaRicorsiva(a, k, sinistra, c-1)
\ELSE
    \RETURN RicercaBinariaRicorsiva(a, k, c+1, destra)
\ENDIF
\end{ALC@g}
\end{algorithmic}
\end{algorithm}

Algoritmo \ref{alg1}:
nota\footnotemark{}.\\
\lipsum[1]

\footnotetext{Forza paolo}
\newpage

\subsection{Esempio di utilizzo}
Trovare l'ultima occorrenza di un elemento $x$ in un array 
ordinato in tempo $O(\log n)$ tramite il paradigma Divide et Impera.

\subsubsection*{Algoritmo:}

\begin{algorithm}[H]
\caption{Ultima occorrenza}
\label{alg2}
\begin{algorithmic}[1]
\STATE UltimaOccorrenza(a, l, r, x):
\begin{ALC@g}
\IF{l $>$ r}
    \RETURN -1
\ENDIF
\IF{l == r $\land$ a[l] == x}
    \RETURN l
\ELSE
    \RETURN -1
\ENDIF
\STATE c = (l + r) / 2
\IF{a[c] $\leq$ x}
    \RETURN $\max$(c, UltimaOccorrenza(a, c + 1, r, x))
\ELSE
    \RETURN UltimaOccorrenza(a, l, c - 1, x)
\ENDIF
\end{ALC@g}
\end{algorithmic}
\end{algorithm}

Algoritmo \ref{alg2}:
nota\footnotemark{}.\\
\lipsum[2]

\footnotetext{Forzaa paolo}
\newpage
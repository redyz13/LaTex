\section{Interval Scheduling Pesato}
Risoluzione del problema dell'Interval Scheduling Pesato.

\paragraph{Relazione di ricorrenza:}

$$\mathit{OPT}(j) =
\begin{cases}
    0 & \text{se $j = 0$ }\\
    \max(v_j + \mathit{OPT}(p(j)), \mathit{OPT}(j-1)) & \text{altrimenti}
\end{cases} $$

\subsubsection*{Esecuzione esterna:}

\begin{algorithmic}[1]
\STATE Input: $\mathrm{n, s_1, \dots, s_n, f_1, \dots, f_n, v_1, \dots, v_n}$
\STATE Ordina i job per tempi di fine in modo che $\mathrm{f_1 \le f_2 \le \dots \le f_n}$
\STATE Calcola $\mathrm{p(1), p(2), \dots, p(n)}$
\FOR{j = 1 to n}
    \STATE{M[j] = \O} \COMMENT{Array globale}
\ENDFOR
\end{algorithmic}

\subsubsection*{Algoritmo:}

\begin{algorithm}[H]
\caption{M-Compute-Opt}
\label{alg3}
\begin{algorithmic}[1]
\STATE M-Compute-Opt(j):
\begin{ALC@g}
\IF{j = 0}
\RETURN 0
\ENDIF
\IF{M[j] = \O}
\STATE M[j] = $\mathrm{\max(v_j + \text{M-Compute-Opt}(p(j)),\
\text{M-Compute-Opt}(j - 1))}$
\ENDIF
\RETURN M[j]
\end{ALC@g}
\end{algorithmic}
\end{algorithm}

Algoritmo \ref{alg3}:
nota\footnotemark{}.\\

\begin{theorem}[Correttezza di M-Compute-Opt]
\label{the1}
L'algoritmo computa correttamente OPT(j)
\end{theorem}

\begin{proof}
Per induzione.
\medbreak

Caso base $j = 0$. Il valore restituito è correttamente 0.
\medbreak

Passo induttivo. Consideriamo un certo $j > 0$ e supponiamo (ipotesi induttiva) 
che l'algoritmo produca il valore corretto di $\mathit{OPT}(i)$ per ogni $i < j$.
Il valore computato per $j$ dall'algoritmo è:
\bigbreak

M-Compute-Opt($j$) = $\max(v_j + \text{M-Compute-Opt}(p(j)),\
\text{M-Compute-Opt}(j - 1))$
\bigbreak

Siccome per ipotesi induttiva:
\begin{itemize}
    \item Valore computato da M-Compute-Opt($p(j$)) = $\mathit{OPT}(p(j))$ e
    \item Valore computato da M-Compute-Opt($j - 1$) = $\mathit{OPT}(j - 1)$
\end{itemize}
\newpage

Allora ne consegue che:
\bigbreak

M-Compute-Opt[$j$] = $\max(v_j + \mathit{OPT}(p(j)),\
\mathit{OPT}(j - 1)) = \mathit{OPT}(j)$
\end{proof}

\footnotetext{Forzaaa paolo}
\newpage